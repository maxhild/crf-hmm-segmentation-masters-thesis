% ****************************************************************************************************
% classicthesis-config-lualatex.tex
% Optimierte Konfiguration für LuaLaTeX + Biber + Deutsch
% ****************************************************************************************************

% ****************************************************************************************************
% 0. Encoding-Setup für LuaLaTeX - inputenc/fontenc NICHT benötigt!
% LuaLaTeX hat native UTF-8 Unterstützung über fontspec
% ****************************************************************************************************
% Für LuaLaTeX werden inputenc und fontenc NICHT benötigt - auskommentiert:
% \PassOptionsToPackage{utf8}{inputenc}
% \usepackage{inputenc}
% \PassOptionsToPackage{T1}{fontenc}
% \usepackage{fontenc}

% Stattdessen fontspec für moderne Font-Unterstützung
\usepackage{fontspec}
\documentclass{article}
\usepackage{amsmath, amssymb}
\usepackage[margin=2.5cm]{geometry}

% ****************************************************************************************************
% 1. Configure classicthesis for your needs here
% ****************************************************************************************************
\PassOptionsToPackage{
  drafting=true,    % print version information on the bottom of the pages
  tocaligned=false, % the left column of the toc will be aligned (no indentation)
  dottedtoc=false,  % page numbers in ToC flushed right
  eulerchapternumbers=false, % use AMS Euler for chapter font (nur für pdfLaTeX)
  floatperchapter=true,     % numbering per chapter for all floats (i.e., Figure 1.1)
  eulermath=false,  % use awesome Euler fonts for mathematical formulae (nur für pdfLaTeX)
  beramono=true,    % toggle a nice monospaced font (w/ bold)
  palatino=true,    % deactivate standard font for loading another one
  style=classicthesis
}{classicthesis}

% ****************************************************************************************************
% 2. Personal data and user ad-hoc commands (deine Daten hier eintragen)
% ****************************************************************************************************
\newcommand{\myTitle}{Optimierung von framebasierter Segmentierung von Endoskopievideos durch Multilabelklassifikation. Vergleich von Markov Models und Conditional Random Fields.\xspace}
\newcommand{\mySubtitle}{Glättung einer Videosegmentation\xspace}
\newcommand{\myDegree}{Bachelor of Science\xspace}
\newcommand{\myName}{Max Hild\xspace}
\newcommand{\mybirthday}{19.09.1999}
\newcommand{\mybirthtown}{Stuttgart}
\newcommand{\mybirthcountry}{Deutschland}
\newcommand{\myProf}{Prof. Dr. Markus Scholz\xspace}
\newcommand{\myOtherProf}{Dr. Thomas J. Lux\xspace}
\newcommand{\mySupervisor}{Dr. Thomas J. Lux\xspace}

\newcommand{\myFaculty}{Medizinische Fakultät\xspace}
\newcommand{\myDepartment}{Institut für Medizinische Informatik, Statistik und Epidemiologie\xspace}
\newcommand{\myUni}{Universität Leipzig\xspace}
\newcommand{\mySubject}{Medizininformatik\xspace}
% Alternative für neuen Master:
% \newcommand{\mySubject}{Medizininformatik M. Sc.\xspace}

\newcommand{\myLocation}{Leipzig\xspace}
\newcommand{\myTime}{\today\xspace} % oder festes Datum
\newcommand{\myVersion}{\classicthesis}

% ********************************************************************
% Setup, finetuning, and useful commands
% ********************************************************************
\providecommand{\mLyX}{L\kern-.1667em\lower.25em\hbox{Y}\kern-.125emX\@}
\newcommand{\ie}{d.\,h.}  % deutsch: das heißt
\newcommand{\Ie}{D.\,h.}
\newcommand{\eg}{z.\,B.}  % deutsch: zum Beispiel  
\newcommand{\Eg}{Z.\,B.}

% ****************************************************************************************************
% 3. Loading packages - angepasst für LuaLaTeX
% ****************************************************************************************************

% ********************************************************************
% Language settings - Deutsch als Hauptsprache
% ********************************************************************
\PassOptionsToPackage{english,ngerman}{babel} % ngerman als letzte = Hauptsprache
\usepackage{babel}

% ********************************************************************
% Font setup für LuaLaTeX mit fontspec
% ********************************************************************
% Moderne Fonts, die gut zu ClassicThesis passen:
\setmainfont{TeX Gyre Pagella}      % Palatino-ähnlich, perfekt für ClassicThesis
\setsansfont{TeX Gyre Heros}        % Helvetica-ähnlich
\setmonofont{TeX Gyre Cursor}       % Courier-ähnlich

% Alternative Fonts (auskommentiert, falls du andere möchtest):
% \setmainfont{Libertinus Serif}    % sehr elegant
% \setmainfont{Latin Modern Roman}  % LaTeX-Standard aber modern
% \setmainfont{Crimson Pro}         % falls installiert

% Math-Fonts für LuaLaTeX
\usepackage{unicode-math}
\setmathfont{TeX Gyre Pagella Math} % passend zu Pagella

% ********************************************************************
% Bibliography mit Biber (empfohlen)
% ********************************************************************
\PassOptionsToPackage{
  backend=biber,        % Biber statt BibTeX
  bibstyle=alphabetic,  % oder numeric, authoryear je nach Präferenz
  citestyle=alphabetic, 
  natbib=true,          % Kompatibilität mit natbib Befehlen (\citep, \citet)
  giveninits=true,      % Vornamen abkürzen
  bibencoding=utf8,     % UTF-8 für Bib-Dateien
  sorting=nyt,          % Name-Jahr-Titel Sortierung
  maxbibnames=10,       % mehr Autorennamen anzeigen
  maxcitenames=2,       % in Zitaten max 2 Namen
  hyperref=true,        % Links in PDF
  backref=true,         % Rückverweise auf Seiten
  isbn=false,           % ISBN ausblenden
  url=false,            % URLs ausblenden (außer für @online)
  doi=true,             % DOI anzeigen
  eprint=false          % arXiv etc. ausblenden
}{biblatex}
\usepackage{biblatex}

% ********************************************************************
% Weitere wichtige Pakete
% ********************************************************************
\PassOptionsToPackage{fleqn}{amsmath}
\usepackage{amsmath}

\usepackage{graphicx}
\usepackage{tabularx}
\usepackage{longtable}
\usepackage{ltxtable}
\usepackage{booktabs}

% Für bessere Mikrotypografie (funktioniert sehr gut mit LuaLaTeX)
\PassOptionsToPackage{
  protrusion=true,
  expansion=true,
  final
}{microtype}
\usepackage{microtype}

% Links und PDF-Features - KORRIGIERT für LuaLaTeX
\PassOptionsToPackage{
  luatex,               % KORRIGIERT: luatex statt pdftex für LuaLaTeX!
  hyperfootnotes=false,
  pdfpagelabels,
  colorlinks=true,
  linkcolor=CTlink,
  citecolor=CTcitation, 
  filecolor=CTlink,
  urlcolor=CTurl
}{hyperref}
\usepackage{hyperref}

% Laden des ClassicThesis-Styles (muss nach hyperref kommen)
\usepackage{classicthesis}

% ********************************************************************
% Weitere nützliche Pakete für wissenschaftliche Arbeiten
% ********************************************************************
\usepackage{acronym}        % für Abkürzungsverzeichnis
\usepackage{listings}       % für Code-Listings
\usepackage{xspace}         % für bessere Leerzeichen in Kommandos
\usepackage{subcaption}     % für Unter-Abbildungen

% FEHLENDE PAKETE HINZUGEFÜGT:
\usepackage{tabulary}       % für tabulary-Umgebung
\usepackage{cleveref}       % für \cref-Kommandos
\usepackage{amsthm}         % für theorem-Umgebungen

% Theorem-Umgebungen definieren
\newtheorem{definition}{Definition}[chapter]
\newtheorem{theorem}{Theorem}[chapter]
\newtheorem{lemma}{Lemma}[chapter]
\newtheorem{example}{Beispiel}[chapter]

% Listings-Setup für Code
\lstset{
  basicstyle=\ttfamily\small,
  breaklines=true,
  frame=single,
  numbers=left,
  numberstyle=\tiny,
  captionpos=b,
  showstringspaces=false
}

% Deutsche Anführungszeichen
\usepackage[autostyle=true,german=quotes]{csquotes}

% ****************************************************************************************************
% Ende der LuaLaTeX-optimierten Konfiguration
% ****************************************************************************************************
