%*****************************************
\chapter{Stand der Forschung}\label{ch:relatedWork}
%*****************************************

Diese Abschlussarbeit basiert auf der Publikation von Lux et. al.`\ ' \cite{lux-2023}, in der ein framebasierter Ansatz zur Segmentierung von Endoskopievideos vorgestellt wird. Lux et. al. schlagen vor, assistierte Dokumentation als Ziel der Segmentierung von Endoskopien zu definieren. Das Ziel ist es also die benötigte Zeit für die Dokumentation sowie die Auswertung von Videodaten zu reduzieren. Dieser Fokus weicht ab von der Zielsetzung vieler anderer Publikationen, die sich auf Empfehlungen der European Society of Gastrointestinal Endoscopy (ESGE) sowie der American Society for Gastrointestinal Endoscopy (ASGE) berufen, laut denen eine Erhöhung der Adenomdetektionsrate ein wichtiges Ziel bei der Qualitätssicherung von Endoskopien ist. \cite{Rex_2022}\cite{kaminski-performance-2017} In mehreren randomisierten Studien konnten System zur Detektion von Adenomen erfolgreich getestet werden. Spadiaccini et al geben beispielsweise eine Steigerung der ADR von 5.6\% in Kombination mit einem System zur Detektion von Mukosaexposition an. \cite{SPADACCINI2023244} Budzyn et al fanden durch eine Analyse des Einsatzes von Systemen zur klinischen Adenomdetektionsrate hingegen eine Reduktion der ADR vor. \cite{budzyn2025deskilling} Auch Hann et al wiesen darauf hin, dass der Einsatz der KI basierten Systeme während der Untersuchung zur Polypendetektion auch zu einem Sinken oder zu gleichbleibender Adenomdetektionsrate führen kann. \cite{hann_entscheidungsunterstutzung_2025}  Der pure Fokus der Analyse von Endoskopievideos auf die Entscheidungsunterstützung scheint hierbei nicht zwangsweise zu besseren Ergebnissen zu führen, ein Umstand der laut Hann et al eng mit dem Verlust von Aufmerksamkeit durch das Vertrauen in Prädiktionssysteme zusammenhängen könnte. Sie wiesen zudem darauf hin, dass die Boxen zur Erkennung von Polypen im klinischen Rahmen ablenkend bei der Resektion sein können, ein weiterer Umstand der adressiert werden sollte. Rafner et al riefen in Ihrem Essay zu Deskilling, dem Verlust von Kompetenzen durch Automatisierung, generell dazu auf, den Einsatz von hybrider Intelligenz zu verfolgen um nicht durch Verlust menschlicher Kompetenz schlechtere Ergebnisse zu erreichen. \cite{rafner_2025_deskilling} Aus diesem Standpunkt heraus macht es also Sinn, die Analyse von klinischen Untersuchungen parallel zur Analyse von medizinischen Fachpersonal für einen Abgleich am Ende zu nutzen. Potenziell sehen einer Umfrage zufolge viele Gastroenterologen den Einsatz der KI zudem zum Zwecke der Zusammenfassung und Segmentierung positiv und weitere Forschung in diesem Bereich ist erwünscht. \cite{munzer_content-based_2018} Savides et. al. analysierten zum Beispiel, dass die Triage interventionaler GI-Überweisungen derzeit in amerikanischen Kliniken mit erheblichem Zeitaufwand verbunden ist. \cite{savides2025triaging} Neben sowie während der Triage ist auch die Dokumentation ein erheblicher Zeitfaktor. In Zukunft könnte dieser Aufwand deutlich reduziert werden, wenn KI-basierte Systeme zur automatisierten Bild- oder Frame-Analyse (z.B. aus endoskopischen Videos) zur Unterstützung der Triagierung eingesetzt werden.\newline 
\citet{lux-2023} beschreiben ein Verfahren zur automatisierten Dokumentation von Endoskopievideos. Dabei wird ein auf Deep Learning basierendes System eingesetzt, um relevante Informationen aus den Videos zu extrahieren und in strukturierter Form bereitzustellen. Diese Form ist hierbei eine Liste von Labels, die die zeitlichen Regionen des Videos beschreiben. Das System nutzt Convolutional Neural Networks (CNNs), um verschiedene Merkmale in den Endoskopiebildern zu erkennen, wie z.B. Polypen, Läsionen oder anatomische Strukturen. Die Autoren zeigen, dass das System eine gute Genauigkeit bei der Erkennung relevanter Merkmale aufweist und somit bereits die Effizienz und Qualität der Endoskopiedokumentation verbessern kann. Probleme bestehen

Das Ziel der Segmentierung ist es, eine Methode zur assistierten Dokumentation zu präsentieren, die Labels mit dazugehörigen Frames verknüpft und somit eine strukturierte Übersicht über das Videomaterial bietet. Dies ist besonders relevant für die Nachbereitung von langen Videodateien, wie Endoskopien. So kann bei der Analyse so schneller auf wichtige Informationen zugreifen können. Viele Autoren betonen, dass eine solche assistierte Dokumentation nicht nur die Effizienz steigert, sondern auch die Genauigkeit der Berichte verbessert, indem menschliche Fehler ausgeglichen werden. Die weitere Entwicklung der assistierten Dokumentation wird als neuer Fokus vorgeschlagen. Diese Empfehlung deckt sich weitestgehend mit der einer weiteren Zielsetzung der  aus \cite{kaminski-performance-2017}, die in ihrer Publikation Forschungsfragen zur Verbesserung der Qualitätssicherung bei der Bewertung der Vollständigkeit von Darmspiegelungen durch Dokumentation aufstellen. Hierbei kann also als Optimierung angenommen werden, dass sich diese vor entweder durch Verbesserung von korrekter Segmentierung der Frames eines Videos oder durch eine Beschleunigung der Verarbeitung und das Nutzbar machen der automatisierten Dokumentation durch Beschleunigung erreichen lässt. Aus diesem Grund möchte ich mich in dieser Arbeit dem von Lux et al genannten Fokus auf assistierte Dokumentation anschließen und eine zeiteffiziente Methode zur Kombination verschiedner Auswertungsquellen zu einem System zur Unterstützung der Frameauswahl und Segmentierung basierend auf Hidden Markov Models präsentieren. 
 %#TODO% Referenzen hinzufügen

In diesem Abschnitt sollen die verschiedenen Methoden die bisher speziell im Bereich der Endoskopie häufig zum Einsatz kommen Erwähnung finden und später im Rahmen der von Lux et.al. geforderten praxisrelevanten assistierten Dokumentation zu einem zusammenhängenden System kombiniert werden.

\subsection{Kombination verschiedener Features wie Farbraumnalyse oder Bewegungserkennung in einem Hidden Markov Model}

Die Klassifikation von Kapselendoskopien mittels Farbanalyse wurde bereits erfolgreich erprobt \cite{4530644} und zeigt gute Ergebnisse. In der Publikation von Mackiewitz et.al. konnte durch die Verwendung von Farbraumklasifizierung und eines HMM gute Ergebnisse bei der automatischen Erkennung der aktuellen Position der endoskopischen Kapsel innerhalb des Körpers gezeigt werden. Um das Rauschen einzelner Features zu kompensieren nutzten die Autoren hierbei eine Kombination aus der Likelihood und Die Autoren der Publikation konstruierten ein Hidden Markov Model, das die Position der Kapsel innerhalb der Endoskopie mit Hilfe einer durch mehrere Trainingsdurchläufe erarbeiteten HMM Übergangsmatrix beschrieb. Die Autoren hoben hierbei hervor, dass die Erkennung der Unterschiede der verschiedenen Organe im Körper wärhend des Durchlaufs der Kapseldoskopie (z.b Übergang von Magen zum Dünndarm zum Dickdarm) den in der Publikation betrachteten Systemen größere Probleme bereitete als die Erkennung des Übergangs vom inneren des Körpers zu den Regionen außerhalb, und damit dem Beginn der Untersuchung. Hier war das sinnvollste System die Kombination eines SVM


\section{Computer Vision - Verarbeitung von Bilddaten}

Der visuelle Kortex des Menschen ist dafür ausgelegt, die Intensitätswerte von Wellenlängen des Lichts, das auf die einzelnen Zellen der Netzhaut fällt, auszuwerten. Auch Computer nutzen die Farbwerte von Pixeln sowie deren Intensität um Bilder zu analysieren. 


\section{CNN - Konvolutionskernels zur Analyse von Bilddaten.}

Lux et al \cite{lux-2023} stellten in ihrer Publikation eine Lösung zur Segmentierung von verschiedenen Bereichen eines Videos vor. Die Autoren verwenden hier ein CNN um einzelne Frames verschiedenen Segmentierungskategorien zuzuordnen. Die Klassifikation von Videos und Bildern aus der Endoskopie mit ResNet, einem residuellen neuronalen Netz, sowie mit anderen CNN wird 
als Methode in vielen Publikationen beschrieben und zeigt generell gute Ergebnisse. \cite{9360830}
CNNs, konvolutionale neuronale Netze, sind eine spezielle Klasse künstlicher neuronaler Netze, die insbesondere für die Verarbeitung von Bilddaten entwickelt wurden. Das zentrale Element eines CNN ist die Faltungsschicht, in der sogenannte Filter (oder Konvolutionskerne) die Pixel des Eingabebilds in lokale Merkmale übersetzen. Dabei gleiten Konvolutionskerne  über das Bild. Mathematisch basiert diese Operation auf der Faltung (engl. convolution), bei der die Werte eines kleinen Bereichs des Bildes mit den Werten des Filters multipliziert und aufsummiert werden. Dadurch entstehen Featuremaps, die charakteristische Muster wie Kanten, Ecken oder komplexere Strukturen erkennen können.

Durch die hierarchische Anordnung mehrerer Faltungs- und Pooling-Schichten können CNNs zunehmend abstrakte Merkmale aus den Rohdaten extrahieren. Dies ermöglicht es, Bildinhalte effizient zu analysieren und zu klassifizieren. Die Parameter der Filter werden während des Trainingsprozesses durch Backpropagation optimiert, sodass das Netzwerk lernt, relevante Merkmale für die jeweilige Aufgabe zu erkennen. 

Um die Qualität der Klassifikation von Bildern bei komputational vertretbarem Rechenaufwand zu verbessern, konnte mit dem InceptionNext ein neuer vielversprechender Ansatz gefunden werden. \cite{yu2024inceptionnext}
Yu et. al. nutzen in Ihrer Publikation kleinere Konvolutionskernels um die Rechenzeit zu reduzieren. Netzwerke wie EfficientNet sind für den mobilen Gebrauch optimiert und können in Echtzeitanalysen eingesetzt werden, eine Eigenschaft die nun dafür genutzt werden kann, die Anwendungen in Echtzeit auszuführen. \cite{Tan2019EfficientNetRM}

\subsection{Vision Transformer - Nutzung von Attentionmechanismen für die Bildklassifikation}

Seitdem die Transformer Architektur \cite{NIPS2017_3f5ee243} den Attentionmechanismus als zentrales Element für die natürliche Sprachverarbeitung einführte, gab es im Bereich der großen Sprachmodelle viele Fortschritte. Das Prinzip des Transformers wird seitdem stetig weiterentwickelt, und führte zu herausragenden Fortschritten bei der Textverarbeitung. Seit der Publikation des Vision Transformer wird der Attentionmechanismus, ein Mechanismus der Analyse von Kovarianzen innerhalb der Trainingsdaten, auch in der Bildverarbeitung eingesetzt \cite{vit_2025}.

\begin{equation}
    Attention(Q,K,V) = softmax(\frac{QK^T}{\sqrt{d}}*V)
\end{equation}

Auch bei der Segmentierung von Endoskopievideos ist es ein Ziel die Segmentierung der Videos kosteneffizienter zu gestalten. Aus diesem Grund hat sich der Vision Transformer bisher noch nicht vollständig durchgesetzt. Das System der konvolutionalen neuronalen Netze arbeitet bei der Analyse von Features hierbei zuverlässiger als


\subsection{Hough-Transformation}