%************************************************
\chapter{Einleitung}\label{ch:introduction}
%************************************************
Eine Abschlussarbeit ist mit einem Projekt vergleichbar und in der Einleitung wird die Aufgabenstellung in ähnlicher Weise beschrieben wie in dem Strukturplan eines Projektes.
Daher sollte die Einleitung mit Hilfe der \enquote{5-Stufen-Methode zur systematischen Strukturplanung}~\citep{ob}\footnotemark{} vorgenommen werden.
\footnotetext{Das Buch ist für Studierende der Medizininformatik an der Universität Leipzig im Moodle verfügbar.
Schauen Sie in den von ihren belegten Modulen nach.}

\section{Gegenstand}
\item \textbf{Ziel Z1.2:} Implementierung eines Conditional Random Field (CRF) Ansatzes als alternative Nachbearbeitungsmethode
\item \textbf{Ziel Z1.3:} Quantitative Evaluation der Segmentierungsgenauigkeit durch Vergleich mit manuell annotierten Start- und Endzeiten
\end{itemize}

\textbf{Ziele zur Lösung von Problem P2 (Fehlende Berücksichtigung visueller Merkmale):}
\begin{itemize}
\item \textbf{Ziel Z2.1:} Integration von Farbinformationen als zusätzliche Eingabe in die statistischen Modelle
\item \textbf{Ziel Z2.2:} Entwicklung eines Bayesschen Ansatzes zur Kombination von Frame-basierten Vorhersagen und visuellen Merkmalen
\end{itemize}

\textbf{Ziele zur Lösung von Problem P3 (Mangelnde Robustheit gegenüber Störungen):}
\begin{itemize}
\item \textbf{Ziel Z3.1:} Verbesserung der Robustheit gegenüber Lichtreflexionen und Wasserartefakten
\item \textbf{Ziel Z3.2:} Bewahrung der Funktionalität der ursprünglichen Glättungsfunktion bei gleichzeitiger Verbesserung der Genauigkeit
\end{itemize}

\textbf{Übergeordnetes Ziel:}
\begin{itemize}
\item \textbf{Ziel Z4.1:} Systematischer Vergleich von HMM und CRF Ansätzen hinsichtlich Genauigkeit, Robustheit und Recheneffizienz für die Endoskopie-Segmentierung
\end{itemize} in der Einleitung wird die Aufgabenstellung in ähnlicher Weise beschrieben wie in dem Strukturplan eines Projektes.
Daher sollte die Einleitung mit Hilfe der \enquote{5-Stufen-Methode zur systematischen Strukturplanung}~\citep{ob}\footnotemark{} vorgenommen werden.
\footnotetext{Das Buch ist für Studierende der Medizininformatik an der Universität Leipzig im Moodle verfügbar.
Schauen Sie in den von ihnen belegten Modulen nach.}

\section{Gegenstand}


\subsection{Forschungsfragen zur Koloskopiequalität}

Endoskopie ist eine der zentralen Behandlungs- und Diagnosemöglichkeiten in der Therapie von Darmkrebs, 
einer der häufigsten und tödlichsten Krebsarten.\citep{labianca2010-colon-cancer}\footnotemark{} 
Nicht von der Hand zu weisen ist hierbei die zeitaufwändige Durchführung und 
Dokumentation der Überwachung des flächigen Gewebes im Darm im Rahmen von Endoskopien. Aus diesem Grund ist es 
der Gastroenterologie konstant ein Anliegen, die Qualitätssicherung bei Darmspiegelungen zu verbessern. 
Ein zuverlässiges Qualitätsmerkmal hierbei ist es beispielsweise, dass die Darmspiegelung eine vorgeschriebene Zeit dauern soll. 
Dadurch steigt die Wahrscheinlichkeit, dass die Darmspiegelung sorgfältig durchgeführt wurde, was nachweislich zu einer 
erhöhten Anzahl Diagnosen führt. Das liegt daran, dass die Durchführung von Darmspiegelungen sorgfältiger wird, 
wenn die hierfür geltenden Qualitätskriterien eingehalten werden. Die Gesellschaft für Gastroenterologie hat hierzu einige Forschungsfragen herausgegeben~\citep{kaminski_performance_2017}, die den Nutzen eines KI-gestützten Systems zur Unterstützung der Qualitätssicherung bei Darmspiegelungen unterstreichen:

\begin{enumerate}
  \item \textbf{Vorbereitung der Prozedur}
    \begin{itemize}
      \item Welche Art von Intervention verbessert die Rate einer adäquaten Darmvorbereitung?
      \item Wie viel Zeit sollte für Screening- und diagnostische Koloskopien eingeplant werden?
    \end{itemize}

  \item \textbf{Vollständigkeit der Prozedur}
    \begin{itemize}
      \item Wie verhalten sich diagnostische Ausbeute (und Intervallkarzinomrate) in Abhängigkeit von einer steigenden Zökumintubationsrate?
      \item Welchen Nutzen hat die Dokumentation der Zökumintubation nur im schriftlichen Bericht im Vergleich zu einem schriftlichen \emph{und} fotografischen Bericht?
    \end{itemize}

  \item \textbf{Identifikation von Pathologien}
    \begin{itemize}
      \item Welcher Zielwert gilt für die Adenomdetektionsrate?
      \item Welches Leistungsmaß spiegelt die Identifikation von Pathologien außerhalb des CRC-Screening-/Überwachungssettings wider?
    \end{itemize}

  \item \textbf{Management von Pathologien}
    \begin{itemize}
      \item Was ist die verlässlichste und praktikabelste Methode, um die Vollständigkeit der Polypentfernung zu messen?
      \item Wie wirksam sind Zusatztechniken/-skalen (Chromendoskopie, Paris-Klassifikation, Tätowierung von Resektionsstellen) im Management von Pathologien?
    \end{itemize}

  \item \textbf{Komplikationen}
    \begin{itemize}
      \item Was ist die verlässlichste und praktikabelste Methode, um Komplikationsraten zu überwachen?
      \item Trägt die Überwachung dazu bei, Komplikationsraten zu senken?
    \end{itemize}

  \item \textbf{Patientenerfahrung}
    \begin{itemize}
      \item Was ist die verlässlichste und praktikabelste Methode, um die Patientenerfahrung zu erfassen?
      \item Wie kann die Patientenerfahrung bei Koloskopien optimiert werden?
    \end{itemize}

  \item \textbf{Nachsorge}
    \begin{itemize}
      \item Welche optimalen Überwachungsintervalle gelten nach der Entfernung kolorektaler Polypen?
      \item Welchen Effekt hat die Überwachung adäquater post-polypektomischer Überwachungsempfehlungen auf die Adhärenz zu Überwachungskoloskopien?
    \end{itemize}
\end{enumerate}

Der Aufbau und die konstante Verbesserung einer automatisierten Aufbereitung von Videos aus Endoskopien könnte dazu beitragen, speziell in der Frage der Quantifizierung mehr aussagekräftige Daten zu bekommen. Lux et~al.\ präsentierten in ihrer Publikation aus dem Jahre 2023~\citep{lux-2023} eine multilabel KI, die aussagekräftige Labels für oben genannte Forschungsfragen beantwortet und somit die Qualitätssicherung bei Darmspiegelungen vereinfacht.

Die Pipeline basiert auf einem RegNetX800MF-Modell, das die Wahrscheinlichkeit von für die aktuelle Endoskopie wichtigen Labels Frame für Frame vorhersagt. Diese Labels werden verwendet, um das Video in Kategorien zu segmentieren. Die Segmentierung dient anschließend dazu, dem Arzt eine Zusammenfassung des Videos zu liefern.

Ein kritisches Segment, das vom Modell ausgegeben wird, verdient besondere Aufmerksamkeit und stellt die primäre Motivation für diese Studie dar. Die \enquote{Outside}-Vorhersage entspricht der Zökum-Rückzugsrate. Diese Messung ist ein wichtiger Qualitätsindikator für die Qualität der Koloskopie. Sie wird verwendet, um die Zeit zu messen, die das Endoskop benötigt, um den Körper zu verlassen, wobei die Messung am Zökum beginnt. Das Zökum ist als erster Teil des Dickdarms der vorgesehene Ausgangspunkt für Koloskopien. Vom Zökum aus sollte die Zeit, die das Koloskop benötigt, um den Körper zu verlassen, mehr als 6 Minuten betragen~\citep{profanter-2020,leung-2019}.

Um diesen Indikator zu berechnen, können genaue Messungen der Zeit verwendet werden, die das Endoskop benötigt, um den Körper zu verlassen. Dieser Ansatz wurde in mehreren Studien getestet, wie der von Vilmann et~al.\ durchgeführten Studie, die zeigte, dass die Zökum-Rückzugsrate mit der Adenom-Detektionsrate verbunden ist~\citep{vilmann-2022}.

\subsection{Aktuelle Herausforderungen in der Endoskopie-Segmentierung}

Die derzeitige Segmentierungspipeline verwendet eine Glättungsfunktion, um Unterbrechungen in den Vorhersagen zu reduzieren. Diese können auftreten, wenn Lichtreflexionen im Video sichtbar sind oder wenn Wasser vom Endoskop stammt. In der aktuellen Theorie bringen diese Ereignisse im Video das Segmentierungsmodell aufgrund eines nicht erkannten visuellen Elements im Frame durcheinander.

Die zentrale Forschungsfrage dieser Arbeit ist daher zu untersuchen, ob die Einführung von Nachbearbeitungsschritten die Qualität der \enquote{Outside}-Vorhersage verbessert. In Beratung zwischen den beteiligten Forschern wurde die Verwendung von Hidden Markov Models (HMM) und Conditional Random Fields (CRF) in Betracht gezogen.

\begin{itemize}
\item Welche Situation liegt vor und was soll getan werden?
\item Worum geht es eigentlich?
\item In welcher Welt/Domäne oder welchem Arbeitsbereich/-gebiet bewegen wir uns im Rahmen der Arbeit
\end{itemize}
\section{Problemstellung}

Die aktuelle Implementierung der Video-Segmentierung für Endoskopie-Aufnahmen weist mehrere kritische Probleme auf, die eine zuverlässige Qualitätsmessung behindern:

\begin{itemize}
\item Die derzeitige Glättungsfunktion zur Nachbearbeitung der Frame-basierten Vorhersagen ist unzureichend für eine präzise Bestimmung der Segment-Grenzen
\item Unterbrechungen in der \enquote{Outside}-Label-Erkennung führen zu ungenauen Messungen der Zökum-Rückzugsrate
\item Die Anonymität der Patienten wird durch unpräzise Segmentierung gefährdet, da \enquote{Outside}-Frames oft Personen zeigen
\item Lichtreflexionen und Wasserartefakte verursachen fehlerhafte Segmentierungen
\end{itemize}

Konkret werden in dieser Arbeit folgende Probleme gelöst:

\begin{itemize}
\item \textbf{Problem P1: Unzuverlässige Segment-Grenzbestimmung} \\
Die aktuelle Methode zur Glättung der Frame-basierten Vorhersagen mittels gleitendem Durchschnitt führt zu ungenauen Start- und Endzeit-Stempeln für die \enquote{Outside}-Segmente, was die Berechnung der Zökum-Rückzugsrate beeinträchtigt.

\item \textbf{Problem P2: Fehlende Berücksichtigung visueller Merkmale} \\
Die derzeitige Nachbearbeitung berücksichtigt keine zusätzlichen visuellen Informationen (wie Farbinformationen) aus den Bildern, die zur Verbesserung der Segmentierungsgenauigkeit beitragen könnten.

\item \textbf{Problem P3: Mangelnde Robustheit gegenüber Störungen} \\
Das aktuelle System ist anfällig für Störungen durch Lichtreflexionen und Wasserartefakte, die zu falschen Segmentierungsunterbrechungen führen.
\end{itemize}

Was aber sind \enquote{Probleme} und wie kann man sie beschreiben?
Die Probleme stehen im engen Zusammenhang mit den in \cref{sec:zielsetzung} beschriebenen Zielen.

\paragraph{Vermeiden Sie Formulierungen wie \enquote{Es ist nicht bekannt, ob$\ldots$} oder \enquote{Es existiert kein$\ldots$}.}
Solche Formulierungen kehren in der Regel einfach das bereits angedachte Lösungsmodell um und postulieren das Fehlen der angedachten Lösung einfach als Problem.
Das ist ähnlich, als wenn es in der Werbung hieße \blockquote{Wenn Sie das Problem haben, dass Ihnen Aspirin fehlt, dann kaufen Sie doch Aspirin}.
Das Problem wäre dann schon gelöst, wenn Sie Aspirin gekauft haben.
Sinnvoller ist diese Aussage:
\blockquote{Wenn Sie das Problem haben, dass Ihnen der Kopf weh tut, dann kaufen Sie doch Aspirin.}
Es ist also bei der Problembeschreibung erforderlich, sich in die Lage dessen zu versetzen, den man mit der angedachten Lösung \enquote{beglücken} möchte.
Sein Problem ist zu ermitteln und so zu formulieren, dass er/sie das Problem wiedererkennt und dadurch geneigt ist, sich für die Lösung des Problems zu interessieren.
Bei der zweiten Problembeschreibung wäre das Problem im Übrigen erst gelöst, wenn die Kopfschmerzen weg sind.


Hierzu noch ein Beispiel aus einer wissenschaftlichen Arbeit:
\blockquote{Problem: Es ist keine ganzheitliche Vorgehensweise bekannt, die die Entwicklung und Verbesserung von Software zur Steigerung der Motivation durch Unterhaltung in der Therapie auf strukturierte Art und Weise unterstützt.}
Als Ziel wird dort formuliert:
\blockquote{Erstellung einer ganzheitlichen Vorgehensweise zur Entwicklung von Software zur Steigerung der Motivation durch Unterhaltung in der Therapie.}
Es ist völlig unklar, wer das Problem hat, dass ihm oder ihr keine Vorgehensweise bekannt ist.
Außerdem wäre das Problem schon gelöst, wenn irgendwie eine ganzheitliche Vorgehensweise bekannt würde.
Tatsächlich schließt diese Arbeit auch einfach mit der Präsentation einer einheitlichen Vorgehensweise und das geschilderte Problem ist gelöst ohne noch weiter zu untersuchen, ob die Vorgehensweise irgendeinen Nutzen erbringt.
Vermutlich liegt das Problem aber etwas tiefer.
Es scheint doch wohl so zu sein, das bislang völlig unbrauchbare Software entwickelt wurde und man die Hoffnung hat, durch eine ganzheitliche Vorgehensweise bessere Ergebnisse erzielen zu können.
Das Problem sollte also eher so formuliert werden:
\blockquote{Problem: Wie aktuelle Publikationen zeigen [12-17], erfüllen die zur Zeit am Markt angebotenen  Softwareprodukte zur Steigerung der Motivation durch Unterhaltung in der Therapie nicht ihren Zweck.
So wird zum Beispiel der Unterhaltungswert von 73\% der Anwender als gering eingestuft und eine Motivationssteigerung für die Therapie konnte in keiner Studie nachgewiesen werden.}
Dazu würde folgendes Ziel passen:
\blockquote{Ziel ist eine ganzheitlichen Vorgehensweise zur Entwicklung von Software zur Steigerung der Motivation durch Unterhaltung in der Therapie, durch deren Anwendung der Unterhaltungswert für die Anwender und eine Motivationssteigerung für die Therapie erreicht werden kann.}
Das hier geschilderte Problem ist jetzt ein Problem von Patient:innen.
Bei diesem Problem und diesem Ziel muss die Arbeit daher damit schließen, dass zumindest an einem Beispiel gezeigt wird, dass die neue Vorgehensweise tatsächlich zu gesteigertem Unterhaltungswert und zur Motivationssteigerung bei Patient:innen führt.
Man könnte aber auch ‚die Latte etwas niedriger hängen‘ und sich zunächst mit folgendem Problem befassen:
\blockquote{Problem: Bei der Entwicklung von Softwareprodukten zur Steigerung der Motivation durch Unterhaltung in der Therapie übersehen viele Entwickler:innen geeignete Möglichkeiten und Spielelemente, die zur Steigerung der Motivation von Patienten eingesetzt werden können oder setzen Spielelemente für die falschen Zwecke oder an der falschen Stelle ein.}
Dazu würde dann folgendes Ziel passen:
\blockquote{Ziel ist eine ganzheitlichen Vorgehensweise zur Entwicklung von Software zur Steigerung der Motivation durch Unterhaltung in der Therapie, die die Softwarentwickler:in ausgehend von den zu erreichenden Zielen systematisch bei der Auswahl geeigneter Spielelemente zur Steigerung von Motivation unterstützt.}
In diesem Fall wird ein Problem vom Softwareentwickler:innen beschrieben und das angestrebte Ziel wird auch diese Entwickler:innen unterstützen.
Hier müsste die Arbeit damit schließen, dass man zeigt, dass es bei dem Entwickeln der Software leichter wird, Elemente so einzusetzen, dass sie dem intendierten Zweck dienen.

\section{Motivation}

Die Lösung der beschriebenen Probleme ist aus mehreren Gründen von hoher Relevanz:

\begin{itemize}
\item \textbf{Medizinische Qualitätssicherung:} Eine präzise Messung der Zökum-Rückzugsrate ist essentiell für die Qualitätsbewertung von Koloskopien und kann direkt zur Verbesserung der Patientenversorgung beitragen.

\item \textbf{Effizienzsteigerung in der Endoskopie:} Automatisierte und zuverlässige Segmentierung kann Ärzte bei der Dokumentation entlasten und die Durchführung von Qualitätskontrollen beschleunigen.

\item \textbf{Datenschutz und Anonymisierung:} Präzise Segmentierung der \enquote{Outside}-Bereiche ist wichtig für die Anonymisierung von Endoskopie-Videos, da diese Bereiche oft Personen zeigen.

\item \textbf{Wissenschaftlicher Fortschritt:} Der Vergleich von Hidden Markov Models und Conditional Random Fields für die Endoskopie-Segmentierung stellt einen wichtigen Beitrag zur medizinischen Informatik dar.
\end{itemize}

\textbf{Nutznießer dieser Arbeit:}
\begin{itemize}
\item Gastroenterologen und Endoskopie-Personal profitieren von verbesserten Qualitätsmetriken
\item Patienten erhalten durch bessere Qualitätskontrolle eine sicherere Behandlung  
\item Die Forschungsgruppe ColoReg erhält optimierte Algorithmen für ihre Segmentierungspipeline
\item Die medizinische Informatik-Community profitiert von methodischen Erkenntnissen zu statistischen Nachbearbeitungsverfahren
\end{itemize}

Die steigenden Zahlen von Darmkrebs-Erkrankungen~\citep{acs-2024,ferlay-2024} und die zunehmende Anzahl von Endoskopie-Videos durch neue Methoden wie die Kapsel-Endoskopie~\citep{koulaouzidis-2021} unterstreichen die Dringlichkeit, automatisierte Qualitätskontrollsysteme zu verbessern.

Ohne ausreichende Gegenstands-, Problem- und Motivationsbeschreibung kann eine Leser:in nicht verstehen, warum z.\,B. eine entwickelte Software sinnvoll ist bzw. zur Lösung welches Problems sie verwendet werden soll.
Gerade für die Medizinische Informatik als eine problemorientierte Disziplin ergibt sich der Wert einer Lösung, z.\,B. einer Software, aber vor allem daraus, ob bzw. wie weit sie ein Problem löst.

Für die Autor:in bedeutet daher eine unzureichende Gegenstands-, Problem- und Motivationsbeschreibung die Gefahr, dass sie oder er sich die zu lösende Problematik nicht ausreichend klar gemacht hat.
Bei der Erstellung der Arbeit besteht dann die Gefahr, dass man möglicherweise methodisch aufregende Lösungen entwirft und realisiert, für die aber ein Problem gar nicht besteht oder die für die Lösung der tatsächlichen Probleme nicht geeignet sind.
Trotz einer möglicherweise brillanten Lösung wäre dann doch eine schlechte Bewertung der Lösung und damit der Arbeit zu erwarten.
Außerdem sollte sich -- auch bei einer Abschlussarbeit -- die Arbeit auch lohnen, d.h. es sollte genügend Motivation geben, viel Zeit und Energie zu investieren.
Aus diesem Grund sollten die Kapitel 1.1 bis 1.3 ausführlich sein.
Ein Umfang von weniger als drei Seiten wird in der Regel nicht ausreichen.
Mit einem Augenzwinkern hier noch Motivationen, die wir nicht gerne in einer Abschlussarbeit sehen:\\
~~\\

\begin{tabulary}{0.965\textwidth}{LL}
Warum lohnt es sich, die genannten Probleme zu lösen?						&\enquote{Weil ich mein Studium endlich hinter mir haben will}, \enquote{Damit ich eine gute Note habe.}, \enquote{Weil Prof. Winter/mein Betreuer es so will.}\\
Wer wird welchen Nutzen von dieser Abschlussarbeit haben?					&\enquote{Ich, weil ich dann endlich mit studieren fertig bin / in den Master darf.}\\
Warum ist die Arbeit wichtig?												&\enquote{Weil mein Betreuer es möchte.}, \enquote{Weil es X noch nicht gibt.}, \enquote{Weil es X nur mit Technik Y gibt, aber nicht mit Z}, \enquote{Weil die Vorarbeiten so schlecht sind}\\
Wer wartet sehnlichst auf die Fertigstellung der Arbeit?					&\enquote{Ich / Mein Betreuer / Prof. Winter / meine Oma} (außer es ist eine Eigenentwicklung für speziell diese Personen)\\
\end{tabulary}

\section{Zielsetzung}\label{sec:zielsetzung}

\begin{itemize}
\item Welche Ergebnisse werden mit dieser Abschlussarbeit angestrebt und welche der o.\,g. Probleme sollen damit jeweils gelöst werden?
\end{itemize}
Bitte jedes Ziel kurz oder ggf. mit Stichworten beschreiben:
\begin{itemize}
\item Ziel(e)/angestrebte(s) Ergebnis(se) zur Lösung von Problem P1:
	\begin{itemize}
	\item Ziel Z1.1: ....
	\item Ziel Z1.2: ....
	\end{itemize}
\end{itemize}

\section{Aufgabenstellung}

\begin{itemize}
\item Wie sollen die o.\,g. Ziele erreicht werden?
\item Was soll zur Erreichung der Ziele bzw. zur Schaffung der Ergebnisse getan werden?
\item Welche Fragen müssen zur Erreichung der Ziele bzw. zur Schaffung der Ergebnisse beantwortet  werden?
\end{itemize}


Bitte geben Sie zu jedem der o.\,g. Ziele mindestens zwei Aufgaben bzw. Fragen an, die bearbeitet bzw. beantwortet werden sollen. Bitte jede Aufgabe bzw. Frage kurz oder ggf. mit Stichworten beschreiben:

\begin{itemize}
\item Aufgaben zu Ziel Z1.1:
	\begin{itemize}
	\item Aufgabe A1.1.1: ....
	\item Aufgabe A1.1.2: ....
	\end{itemize}
\end{itemize}
