\chapter{Literaturverzeichnis-Tipps}

\section{biber}
biber\footnote{\url{https://ctan.org/pkg/biber}} löst das alte BibTeX ab, greift aber genauso auf .bib-Dateien zu, in denen die Literaturquellen definiert sind. In dieser Vorlage ist das alte bibtex-Backend eingestellt, in \texttt{classicthesis-config.tex} kann aber auch das neuere \enquote{biber}-Backend aktiviert werden, wodurch z.B. unicode-Zeichen möglich werden und Umlaute nicht mehr escaped werden müssen.
Eine .bib-Datei besteht aus einer Liste von Einträgen, die z.B. auf \href{https://scholar.google.com}{Google Scholar} gefunden und per copy-paste in die \texttt{thesis.bib} eingefügt werden können.
Diese müssen aber manuell geprüft und bei fehlenden oder inkorrekten Angaben wie Autoren oder dem Verlag korrigiert werden.

\section{Zitieren}
Diese Vorlage ist mit natbib kompatibel, sodass das neben \texttt{cite} auch die Kommandos \texttt{citet} (Zitieren im Text) und \texttt{citep} (Zitieren in Klammern) ablöst.
Details stehen in der \href{http://mirrors.ctan.org/macros/latex/contrib/natbib/natbib.pdf}{natbib-Paketdokumentation}.

\paragraph{Beispielzitierungen}
\begin{itemize}
\item \citet{sniktec} untersuchen verschiedene Verfahren für X.
\item X ist \enquote{ein gutes Verfahren, um Y auf Basis von Z zu generieren}~\citep{sniktec}.\footnote{Wörtliches Zitat}
\item In dieser Situation bietet sich auch X an, da es ein gutes Verfahren ist, um mit Z Y zu erstellen~\citep[vgl.]{sniktec}.\footnote{Sinngemäßes Zitat}
\end{itemize}


\subsection{Häufige Fehler}
\begin{itemize}
\item Zitierung in Fußnote
\item Verwechseln von wörtlichem Zitat (Anführungszeichen), sinngemäßem Zitat (\enquote{vgl.}) und anderer Aussage (weder Anführungszeichen noch \enquote{vgl.})
\item "Falsche Anführungszeichen" (immer \texttt{enquote} benutzen, dann kann dieser Fehler nicht passieren)
\end{itemize}

\section{Zitierstile}
Bei einer Publikation ist der Zitierstil normalerweise vorgegeben, aber bei unseren Abschlussarbeiten gibt es keine Vorgabe.
Ein numerischer Zitierstil ist sehr kompakt und bietet sich daher an, wenn ein Seitenlimit eingehalten werden muss, aber für eine Abschlussarbeit empfehlen wir einen Autor-Jahr-Zitierstil (in dieser Vorlage bereits eingestellt), da dieser mehr Informationen vermittelt.
