% !TeX program = lualatex
% !BIB program = biber
\documentclass[11pt]{scrbook}

% Sprache & Typografie
\usepackage[ngerman]{babel}
\usepackage[autostyle]{csquotes}

% biber (Autor-Jahr), biber, deutsche Lokalisierung
\usepackage[
  backend=biber,
  style=authoryear-icomp,
  sorting=nyt
]{biber}

% Literaturdatenbank einbinden
\addbibresource{Bibliography.bib} % <--- ggf. anpassen

% LuaLaTeX: moderne Schriftwahl (optional)
\usepackage{fontspec}
\setmainfont{TeX Gyre Pagella}

\begin{document}
\tableofcontents

\chapter{Zitierbeispiele (biber, deutsch)}

% ------------------------------------------------------
\section{Direktes Zitat (wörtlich)}
% kurzer Inline-Quote mit Seitenzahl:
\enquote{Kurzes wörtliches Zitat} \parencite[15]{muster2020}.

% längeres Blockzitat (csquotes übernimmt Anführungen/Einzug), Quelle danach:
\blockquote{Dies ist ein längeres wörtliches Zitat, das als Blocksatz gesetzt wird.}
\parencite[78]{muster2020}

% Text-gebundenes (Subjekt = Autor), mit Seitenangabe:
\textcite[22]{muster2020} argumentiert, dass \enquote{…}.

% ------------------------------------------------------
\section{Indirektes Zitat (paraphrasiert)}
% vgl. als Prenote, Seitenangabe als Postnote:
Vgl.\ \parencite[vgl.][33]{beispiel2019} für eine alternative Definition.

% Textgebunden (Autor im Satz):
Wie \textcite[Kap.~3]{beispiel2019} zeigen, ist die Lage komplex.

% ------------------------------------------------------
\section{Fußnotenzitation}
% Zitat in einer Fußnote, mit Seitenzahl:
Dies wird häufig so gesehen.\footcite[101]{muster2020}

% ------------------------------------------------------
\section{Mehrfachzitate}
% mehrere Quellen auf einmal (Klammer):
Siehe \parencites{muster2020}{beispiel2019}{alpha2018}.

% mehrere mit individuellen Seiten:
Siehe \parencites[15]{muster2020}[33]{beispiel2019}[§2]{alpha2018}.

% Textgebundene Mehrfachzitate:
\textcites{muster2020}{beispiel2019} berichten übereinstimmend, dass ...

% ------------------------------------------------------
\section{Spezielle Angaben: Seiten, Spannen, Kapitel}
% reine Jahres- oder Autorenangaben:
Nur Autor: \citeauthor{muster2020}; nur Jahr: \citeyear{muster2020}.

% Seitenbereich:
Siehe \parencite[15--18]{muster2020}.

% Kapitel/Paragraf:
Siehe \parencite[Kap.~4]{muster2020}; oder \parencite[§\,2]{alpha2018}.

% prä- und postnote gleichzeitig:
Siehe \parencite[vgl. auch][Anhang~B]{muster2020}.

% ------------------------------------------------------
\section{Direkte Zitate mit Quellenangabe im Satzfluss}
% Autor im Text, Jahr/Seite in Klammern:
Nach \textcite[45]{muster2020} ist das Verfahren robust.
% Nur Klammer (Autor, Jahr, Seite):
Aktuelle Befunde bestätigen dies \parencite[45]{muster2020}.

% ------------------------------------------------------
\section{Literaturverzeichnis}
\printbibliography

\end{document}
