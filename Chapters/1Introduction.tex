%************************************************
\chapter{Einleitung}\label{ch:introduction}
%************************************************
Die Durchführung von Endoskopien ist ein wesentlicher Bestandteil der modernen medizinischen Diagnostik. Mit dem Aufkommen von Künstlicher Intelligenz (KI) und maschinellem Lernen eröffnen sich neue Möglichkeiten zur Automatisierung und Verbesserung der Analyse von Endoskopie-Videos. Diese Arbeit konzentriert sich auf die Optimierung der framebasierten Segmentierung von Endoskopie-Videos durch den Vergleich von Markov Models und Conditional Random Fields.

\section{Gegenstand}

\subsection{Endoskopie als diagnostisches Werkzeug}

Endoskopie ist ein weit verbreitetes diagnostisches Werkzeug in der Medizin. Eine Anwendung der Koloskopie (Endoskopie im Dickdarm) liegt im Screening nach Adenomen. Dies ist besonders relevant für die Behandlung von Patienten mit bekannten Risikofaktoren wie entzündlichen Darmerkrankungen~\citep{vleggaar-2007}. Diese Patienten werden präventiv in Routineuntersuchungen gescreent~\citep{arnold-2020}.

Aus diesem Grund veröffentlichen Verbände wie die Deutsche Gesellschaft für Viszeralmedizin regelmäßige Richtlinien zur Anzahl der durchgeführten Endoskopien. Um sich als Exzellenzzentrum zu qualifizieren, muss ein Krankenhaus in Deutschland mindestens 2500 endoluminale Endoskopien durchführen~\citep{dgav-ev-2015}.

Mit dem Aufkommen kostengünstigerer Methoden wie der Kapsel-Endoskopie wird erwartet, dass die Anzahl der zu analysierenden Endoskopie-Videos steigen wird~\citep{koulaouzidis-2021}\citep{8903282}. Diese Zahlen, kombiniert mit der drohenden Gefahr verschiedener kolorektaler Krebsarten in steigenden Zahlen~\citep{acs-2024,ferlay-2024}, unterstreichen die Notwendigkeit weiterer Forschung zur Verbesserung des Screening-Prozesses.

\subsection{KI-gestützte Segmentierung in der Endoskopie}

Die Verbesserung des Endoskopie-Workflows könnte in Zukunft durch KI-basierte Segmentierung für schnelleres Screening und Dokumentation erfolgen. Hier haben Lux et~al.\ eine nutzbare Segmentierungspipeline in einer früheren Publikation demonstriert~\citep{lux-2023}.

Die Pipeline basiert auf einer Multi-Label-KI, die die Wahrscheinlichkeit von für die aktuelle Endoskopie wichtigen Labels vorhersagt. Die Labels werden verwendet, um das Video in Kategorien zu segmentieren. Ein fein abgestimmtes RegNetX800MF-Modell sagt die Wahrscheinlichkeit von für den aktuellen Endoskopie-Frame wichtigen Labels Frame für Frame voraus. Die Labels werden verwendet, um das Video in Kategorien zu segmentieren. Die Segmentierung wird dann verwendet, um dem Arzt eine Zusammenfassung des Videos zu liefern.

\subsection{Die Bedeutung der Zökum-Rückzugsrate}

Ein kritisches Segment, das vom Modell ausgegeben wird, verdient besondere Aufmerksamkeit und ist die primäre Motivation für diese Studie. Die \enquote{Outside}-Vorhersage entspricht der Zökum-Rückzugsrate. Diese Messung ist ein wichtiger Qualitätsindikator für die Qualität der Koloskopie. Sie wird verwendet, um die Zeit zu messen, die das Endoskop benötigt, um den Körper zu verlassen, wobei die Messung am Zökum beginnt. 

Das Zökum ist als erster Teil des Dickdarms der vorgesehene Ausgangspunkt für Koloskopien. Vom Zökum aus sollte die Zeit, die das Koloskop benötigt, um den Körper zu verlassen, mehr als 6 Minuten betragen~\citep{profanter-2020,leung-2019}. Um diesen Indikator zu berechnen, können genaue Messungen der Zeit verwendet werden, die das Endoskop benötigt, um den Körper zu verlassen.

Dieser Ansatz wurde in mehreren Studien getestet, wie der von Vilmann et~al.\ durchgeführten Studie, die zeigte, dass die Zökum-Rückzugsrate mit der Adenom-Detektionsrate verbunden ist~\citep{vilmann-2022}.

\subsection{Aktuelle Herausforderungen}

Die derzeitige Segmentierungspipeline verwendet eine Glättungsfunktion zur Nachbearbeitung der Frame-basierten Vorhersagen. Der Hauptgrund für die Einführung einer Glättungsfunktion ist, dass die Segmente falsch unterbrochen werden können. Dies kann beispielsweise passieren, wenn Lichtreflexionen im Video sichtbar sind oder wenn der Wasserjet des Endoskops im Bild für Reflexionen sorgt.
Diese Probleme treten insbesondere bei Ein- und Autritt des Endoskops in den Körper auf, da sich hier die Lichtverhältnisse rapide ändern.

In der aktuellen Theorie bringen diese Ereignisse im Video das Segmentierungsmodell aufgrund eines nicht erkannten visuellen Elements im Frame durcheinander. Aus diesem Grund sind die Farbinformationen aus dem Bild ein möglicher Kandidat als zusätzliche Eingabe für ein Bayessches statistisches Modell zur Glättung der Vorhersagen.

\section{Problemstellung}

Die aktuelle Implementierung der Video-Segmentierung für Endoskopie-Aufnahmen weist mehrere kritische Probleme auf, die eine zuverlässige Qualitätsmessung behindern:

\begin{itemize}
\item Die derzeitige Glättungsfunktion zur Nachbearbeitung der Frame-basierten Vorhersagen ist unzureichend für eine präzise Bestimmung der Segment-Grenzen
\item Unterbrechungen in der \enquote{Outside}-Label-Erkennung führen zu ungenauen Messungen der Zökum-Rückzugsrate
\item Die Anonymität der Patienten wird durch unpräzise Segmentierung gefährdet, da \enquote{Outside}-Frames oft Personen zeigen
\item Lichtreflexionen und Wasserartefakte verursachen fehlerhafte Segmentierungen
\end{itemize}

Konkret werden in dieser Arbeit folgende Probleme gelöst:

\begin{itemize}
\item \textbf{Problem P1: Unzuverlässige Segment-Grenzbestimmung} \\
Die aktuelle Methode zur Glättung der Frame-basierten Vorhersagen mittels gleitendem Durchschnitt führt zu ungenauen Start- und Endzeit-Stempeln für die \enquote{Outside}-Segmente, was die Berechnung der Zökum-Rückzugsrate beeinträchtigt.

\item \textbf{Problem P2: Fehlende Berücksichtigung visueller Merkmale} \\
Die derzeitige Nachbearbeitung berücksichtigt keine zusätzlichen visuellen Informationen (wie Farbinformationen) aus den Bildern, die zur Verbesserung der Segmentierungsgenauigkeit beitragen könnten.

\item \textbf{Problem P3: Mangelnde Robustheit gegenüber Störungen} \\
Das aktuelle System ist anfällig für Störungen durch Lichtreflexionen und Wasserartefakte, die zu falschen Segmentierungsunterbrechungen führen.
\end{itemize}

Diese Probleme stehen im engen Zusammenhang mit den in \cref{sec:zielsetzung} beschriebenen Zielen.

\section{Motivation}

Die Lösung der beschriebenen Probleme ist aus mehreren Gründen von hoher Relevanz:

\begin{itemize}
\item \textbf{Medizinische Qualitätssicherung:} Eine präzise Messung der Zökum-Rückzugsrate ist essentiell für die Qualitätsbewertung von Koloskopien und kann direkt zur Verbesserung der Patientenversorgung beitragen.

\item \textbf{Effizienzsteigerung in der Endoskopie:} Automatisierte und zuverlässige Segmentierung kann Ärzte bei der Dokumentation entlasten und die Durchführung von Qualitätskontrollen beschleunigen.

\item \textbf{Datenschutz und Anonymisierung:} Präzise Segmentierung der \enquote{Outside}-Bereiche ist wichtig für die Anonymisierung von Endoskopie-Videos, da diese Bereiche oft Personen zeigen.

\item \textbf{Wissenschaftlicher Fortschritt:} Der Vergleich von Hidden Markov Models und Conditional Random Fields für die Endoskopie-Segmentierung stellt einen wichtigen Beitrag zur medizinischen Informatik dar.
\end{itemize}

\textbf{Nutznießer dieser Arbeit:}
\begin{itemize}
\item Gastroenterologen und Endoskopie-Personal profitieren von verbesserten Qualitätsmetriken
\item Patienten erhalten durch bessere Qualitätskontrolle eine sicherere Behandlung  
\item Die Forschungsgruppe ColoReg erhält optimierte Algorithmen für ihre Segmentierungspipeline
\item Die medizinische Informatik-Community profitiert von methodischen Erkenntnissen zu statistischen Nachbearbeitungsverfahren
\end{itemize}

Die steigenden Zahlen von Darmkrebs-Erkrankungen~\citep{acs-2024,ferlay-2024} und die zunehmende Anzahl von Endoskopie-Videos durch neue Methoden wie die Kapsel-Endoskopie~\citep{koulaouzidis-2021} unterstreichen die Dringlichkeit, automatisierte Qualitätskontrollsysteme zu verbessern.

\section{Zielsetzung}\label{sec:zielsetzung}

Die Hauptziele dieser Masterarbeit sind die Entwicklung und Evaluation verbesserter Nachbearbeitungsverfahren für die Frame-basierte Endoskopie-Video-Segmentierung.

\textbf{Ziele zur Lösung von Problem P1 (Unzuverlässige Segment-Grenzbestimmung):}
\begin{itemize}
\item \textbf{Ziel Z1.1:} Implementierung eines Hidden Markov Model (HMM) Ansatzes zur Ersetzung der aktuellen Glättungsfunktion
\item \textbf{Ziel Z1.2:} Implementierung eines Conditional Random Field (CRF) Ansatzes als alternative Nachbearbeitungsmethode
\item \textbf{Ziel Z1.3:} Quantitative Evaluation der Segmentierungsgenauigkeit durch Vergleich mit manuell annotierten Start- und Endzeiten
\end{itemize}

\textbf{Ziele zur Lösung von Problem P2 (Fehlende Berücksichtigung visueller Merkmale):}
\begin{itemize}
\item \textbf{Ziel Z2.1:} Integration von Farbinformationen als zusätzliche Eingabe in die statistischen Modelle
\item \textbf{Ziel Z2.2:} Entwicklung eines Bayesschen Ansatzes zur Kombination von Frame-basierten Vorhersagen und visuellen Merkmalen
\end{itemize}

\textbf{Ziele zur Lösung von Problem P3 (Mangelnde Robustheit gegenüber Störungen):}
\begin{itemize}
\item \textbf{Ziel Z3.1:} Verbesserung der Robustheit gegenüber Lichtreflexionen und Wasserartefakten
\item \textbf{Ziel Z3.2:} Bewahrung der Funktionalität der ursprünglichen Glättungsfunktion bei gleichzeitiger Verbesserung der Genauigkeit
\end{itemize}

\textbf{Übergeordnetes Ziel:}
\begin{itemize}
\item \textbf{Ziel Z4.1:} Systematischer Vergleich von HMM und CRF Ansätzen hinsichtlich Genauigkeit, Robustheit und Recheneffizienz für die Endoskopie-Segmentierung
\end{itemize}

\section{Aufgabenstellung}

Die Forschungsfrage dieser Arbeit ist daher zu untersuchen, ob die Einführung von Nachbearbeitungsschritten die Qualität der \enquote{Outside}-Vorhersage verbessert. In diesem Fall wäre die Methode, die aktuelle Glättungsfunktion in der Nachbearbeitung durch einen dieser statistischen Ansätze zu ersetzen.
Der Vollständigkeit halber sollen auch die anderen Labels (wie \enquote{Esophagus}, \enquote{Stomach}, \enquote{Duodenum}, \enquote{Ileum}, \enquote{Colon}, \enquote{Rectum}) in die Analyse einbezogen werden, um die allgemeine Leistungsfähigkeit der Segmentierungspipeline zu bewerten.

\textbf{Aufgaben zur Erreichung der Ziele:}

\textbf{Aufgaben zu Ziel Z1.1 (HMM-Implementierung):}
\begin{itemize}
\item \textbf{Aufgabe A1.1.1:} Analyse der aktuellen Glättungsfunktion und ihrer Schwächen
\item \textbf{Aufgabe A1.1.2:} Design und Implementierung eines HMM-basierten Nachbearbeitungsalgorithmus
\item \textbf{Aufgabe A1.1.3:} Integration des HMM-Ansatzes in die bestehende Segmentierungspipeline
\end{itemize}

\textbf{Aufgaben zu Ziel Z1.2 (CRF-Implementierung):}
\begin{itemize}
\item \textbf{Aufgabe A1.2.1:} Konzeption eines CRF-Modells für die Endoskopie-Segmentierung
\item \textbf{Aufgabe A1.2.2:} Implementierung und Training des CRF-Ansatzes
\item \textbf{Aufgabe A1.2.3:} Optimierung der CRF-Parameter für die Endoskopie-Domäne
\end{itemize}

\textbf{Aufgaben zu Ziel Z2.1 (Integration visueller Merkmale):}
\begin{itemize}
\item \textbf{Aufgabe A2.1.1:} Extraktion und Analyse von Farbmerkmalen aus Endoskopie-Frames
\item \textbf{Aufgabe A2.1.2:} Integration der Farbinformationen in die HMM- und CRF-Modelle
\end{itemize}

\textbf{Aufgaben zu Ziel Z3.1 (Robustheit):}
\begin{itemize}
\item \textbf{Aufgabe A3.1.1:} Identifikation und Charakterisierung von Störungen (Lichtreflexionen, Wasser)
\item \textbf{Aufgabe A3.1.2:} Entwicklung robuster Merkmale zur Störungsunterdrückung
\end{itemize}

\textbf{Aufgaben zu Ziel Z4.1 (Evaluation):}
\begin{itemize}
\item \textbf{Aufgabe A4.1.1:} Systematische Evaluation der HMM- und CRF-Ansätze gegen die ursprüngliche Glättungsfunktion
\item \textbf{Aufgabe A4.1.2:} Vergleich mit manuell annotierten Start- und Endzeiten zur Validierung der Genauigkeit
\item \textbf{Aufgabe A4.1.3:} Bewertung der Recheneffizienz und praktischen Anwendbarkeit
\end{itemize}

Idealerweise würde das Ergebnis darin bestehen, die Start- und Endzeit-Stempel der Segmente zuverlässiger zu setzen. Bei Messungen sollte dies zu Änderungen beim Vergleich der Vorhersage des Modells mit annotierten Start- und Endzeiten sowie mit der ursprünglichen Glättungsfunktion unter Verwendung des gleitenden Durchschnitts der Vorhersagen führen.
